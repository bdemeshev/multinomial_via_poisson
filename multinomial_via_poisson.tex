% arara: xelatex
\documentclass[12pt]{article}

% \usepackage{physics}

\usepackage{hyperref}
\hypersetup{
    colorlinks=true,
    linkcolor=blue,
    filecolor=magenta,      
    urlcolor=cyan,
    pdftitle={Overleaf Example},
    pdfpagemode=FullScreen,
    }

\usepackage{verse}

\usepackage{tikzducks}

\usepackage{tikz} % картинки в tikz
\usetikzlibrary{shapes, arrows, positioning}
\usepackage{microtype} % свешивание пунктуации

\usepackage{array} % для столбцов фиксированной ширины

\usepackage{indentfirst} % отступ в первом параграфе

\usepackage{sectsty} % для центрирования названий частей
\allsectionsfont{\centering}

\usepackage{amsmath, amsfonts, amssymb} % куча стандартных математических плюшек

\usepackage{comment}

\usepackage[top=2cm, left=1.2cm, right=1.2cm, bottom=2cm]{geometry} % размер текста на странице

\usepackage{lastpage} % чтобы узнать номер последней страницы

\usepackage{enumitem} % дополнительные плюшки для списков
%  например \begin{enumerate}[resume] позволяет продолжить нумерацию в новом списке
\usepackage{caption}

\usepackage{url} % to use \url{link to web}




\newcommand{\smallduck}{\begin{tikzpicture}[scale=0.3]
    \duck[
        cape=black,
        hat=black,
        mask=black
    ]
    \end{tikzpicture}}

\usepackage{fancyhdr} % весёлые колонтитулы
\pagestyle{fancy}
\lhead{}
\chead{Мультиномиальная логит-модель и пуассоновский поток}
\rhead{}
\lfoot{}
\cfoot{}
\rfoot{}

\renewcommand{\headrulewidth}{0.4pt}
\renewcommand{\footrulewidth}{0.4pt}

\usepackage{tcolorbox} % рамочки!

\usepackage{todonotes} % для вставки в документ заметок о том, что осталось сделать
% \todo{Здесь надо коэффициенты исправить}
% \missingfigure{Здесь будет Последний день Помпеи}
% \listoftodos - печатает все поставленные \todo'шки


% более красивые таблицы
\usepackage{booktabs}
% заповеди из докупентации:
% 1. Не используйте вертикальные линни
% 2. Не используйте двойные линии
% 3. Единицы измерения - в шапку таблицы
% 4. Не сокращайте .1 вместо 0.1
% 5. Повторяющееся значение повторяйте, а не говорите "то же"


\setcounter{MaxMatrixCols}{20}
% by crazy default pmatrix supports only 10 cols :)


\usepackage{fontspec}
\usepackage{libertine}
\usepackage{polyglossia}


\usepackage[bibencoding = auto,
backend = biber,
sorting = nyt, % name-year-title sorting
style=authoryear]{biblatex}

\addbibresource{multinomial_via_poisson.bib}


\setmainlanguage{russian}
\setotherlanguages{english}

% download "Linux Libertine" fonts:
% http://www.linuxlibertine.org/index.php?id=91&L=1
% \setmainfont{Linux Libertine O} % or Helvetica, Arial, Cambria
% why do we need \newfontfamily:
% http://tex.stackexchange.com/questions/91507/
% \newfontfamily{\cyrillicfonttt}{Linux Libertine O}

\AddEnumerateCounter{\asbuk}{\russian@alph}{щ} % для списков с русскими буквами
% \setlist[enumerate, 2]{label=\asbuk*),ref=\asbuk*}

%% эконометрические сокращения
\DeclareMathOperator{\Cov}{\mathbb{C}ov}
\DeclareMathOperator{\Corr}{\mathbb{C}orr}
\DeclareMathOperator{\Var}{\mathbb{V}ar}
\DeclareMathOperator{\col}{col}
\DeclareMathOperator{\row}{row}
\DeclareMathOperator{\rank}{rank}

\let\P\relax
\DeclareMathOperator{\P}{\mathbb{P}}

\DeclareMathOperator{\E}{\mathbb{E}}
% \DeclareMathOperator{\tr}{trace}
\DeclareMathOperator{\card}{card}
\DeclareMathOperator{\mgf}{mgf}

\DeclareMathOperator{\Convex}{Convex}
\DeclareMathOperator{\plim}{plim}

\usepackage{mathtools}
\DeclarePairedDelimiter{\norm}{\lVert}{\rVert}
\DeclarePairedDelimiter{\abs}{\lvert}{\rvert}
\DeclarePairedDelimiter{\scalp}{\langle}{\rangle}
\DeclarePairedDelimiter{\ceil}{\lceil}{\rceil}

\newcommand{\cN}{\mathcal{N}}
\newcommand{\cF}{\mathcal{F}}

\newcommand{\nendo}{n_{\text{endo}}}
\newcommand{\nexo}{n_{\text{exo}}}

\newcommand{\RR}{\mathbb{R}}
\newcommand{\NN}{\mathbb{N}}
\newcommand{\hb}{\hat{\beta}}

\newcommand{\dBern}{\mathrm{Bern}}
\newcommand{\dPois}{\mathrm{Pois}}
\newcommand{\dBin}{\mathrm{Bin}}
\newcommand{\dMult}{\mathrm{Mult}}
\newcommand{\dGeom}{\mathrm{Geom}}
\newcommand{\dNHGeom}{\mathrm{NHGeom}}
\newcommand{\dHGeom}{\mathrm{HGeom}}
\newcommand{\dDUnif}{\mathrm{DUnif}}
\newcommand{\dFS}{\mathrm{FS}}
\newcommand{\dNBin}{\mathrm{NBin}}

\newcommand{\dTri}{\mathrm{Triangle}}
\newcommand{\dUnif}{\mathrm{Unif}}
\newcommand{\dU}{\mathrm{U}}
\newcommand{\dCauchy}{\mathrm{Cauchy}}
\newcommand{\dN}{\mathcal{N}}
\newcommand{\dLN}{\mathcal{LN}}
\newcommand{\dExpo}{\mathrm{Expo}} % o is probably great to avoid confusion with exp function
\newcommand{\dExp}{\dExpo}
\newcommand{\dBeta}{\mathrm{Beta}}
\newcommand{\dGamma}{\mathrm{Gamma}}
\newcommand{\dWei}{\mathrm{Wei}}
\newcommand{\dLogistic}{\mathrm{Logistic}}
\newcommand{\dRayleigh}{\mathrm{Rayleigh}}
\newcommand{\dPareto}{\mathrm{Pareto}}
\newcommand{\dGumbel}{\mathrm{Gumbel}}




\begin{document}


\begin{verse}
    \begin{flushright}
        ...
    \end{flushright}
\end{verse}

Цель этой заметки — показать связь мультиномильной модели с пуассоновским потоком, а заодно доказать формулу для вероятностей :)


\section*{Равномерное, экспоненциальное и распределение Гумбеля}
Величина $L$ распределена равномерно на отрезке $[0;1]$. 
Посмотрим на цепочку 
\[
L \overset{h}{\rightarrow} M \overset{h}{\rightarrow} R, \quad h(x) = -\ln x,
\] 
где каждая следующая величина получается как минус логарифм предыдущей, $M = -\ln L$, $R = - \ln M$.
В обратную сторону, $M = \exp(-R)$, $L = \exp(- M)$,
\[
    L \overset{g}{\leftarrow} M \overset{g}{\leftarrow} R, \quad g(x) = \exp(-x).
\]
Найдём, как меняется функция плотности через дифференциальную форму.
Стартуем с плотности $f_L(\ell) = 1$ или с формы $f_L(\ell)d\ell = 1\cdot d\ell$.
И просто подставляем обратное преобразование $\ell = \exp(-m)$ в $f_L(\ell)d\ell$,
\[
f_L(\ell)d\ell = 1 \cdot d\ell = 1 \cdot d(\exp(-m)) = -\exp(-m) dm.
\]
Отсюда, $f_M(m)=\exp(-m)$, случайная величина $M$ имеет \emph{экспоненциальное распределение} с единичной интенсивностью. 
Знак минус вовсе не говорит, что плотность отрицательная :) 
Дело лишь в том, что положительное $d\ell$ соответствует отрицательному $dm$. 

Теперь подставляем обратное преобразование $m = \exp(-r)$ в $f_M(m)dm$,
\[
f_M(m)dm = \exp(-m) \cdot dm = \exp(-\exp(-r)) \cdot d(\exp(-r)) = -\exp(-r - \exp(-r)) dr.
\]
Замечаем, что функция плотности величины $R$ равна $f_R(r) = \exp(-r - \exp(-r))$.
Величина $R$ имеет распределение Гумбеля.

Снимем маски с полученной цепочки :)
\[
\dUnif[0; 1] \overset{h}{\rightarrow} \dExpo(\lambda = 1) \overset{h}{\rightarrow} \dGumbel, \quad h(x) = -\ln x.
\]

\section*{Мультиномиальная модель на трёх языках}

Изложим классическую мультиномиальную модель на примере трёх альтернатив на языках трёх распределений.

С помощью распределения Гумбеля:
\[
\begin{cases}
u_{ij} \sim \dGumbel \\
y_{ia}^* = x_i^T \beta_a + u_{ia}, \text{ где } \beta_a = 0 \\
y_{ib}^* = x_i^T \beta_b + u_{ib} \\
y_{ic}^* = x_i^T \beta_c + u_{ic} \\
y_i = \begin{cases}
a, \text{ если } y_{ia}^* = \max\{y_{ia}^*, y_{ib}^*, y_{ic}^*\} \\
b, \text{ если } y_{ib}^* = \max\{y_{ia}^*, y_{ib}^*, y_{ic}^*\} \\
c, \text{ если } y_{ic}^* = \max\{y_{ia}^*, y_{ib}^*, y_{ic}^*\} \\
\end{cases}
\end{cases}
\]


С помощью экспоненциального распределения, $v_{ij} = \exp(-u_{ij})$, $T_{ij} = \exp(-y_{ij}^*)$.
\[
\begin{cases}
v_{ij} \sim \dExpo(\lambda=1) \\
T_{ia} = v_{ia} / \exp(x_i^T \beta_a), \text{ где } \beta_a = 0 \\
T_{ib} = v_{ib} / \exp(x_i^T \beta_b)\\
T_{ic} = v_{ic} / \exp(x_i^T \beta_c)\\
y_i = \begin{cases}
a, \text{ если } T_{ia} = \min\{T_{ia}, T_{ib}, T_{ic}\} \\
b, \text{ если } T_{ib} = \min\{T_{ia}, T_{ib}, T_{ic}\} \\
c, \text{ если } T_{ic} = \min\{T_{ia}, T_{ib}, T_{ic}\} \\
\end{cases}
\end{cases}
\]
Замечаем, что в экспоненциальном изложении мультиномиальной логит-модели $T_{ij} \sim \dExpo(\lambda = \exp(x_i^T \beta_j))$.


С помощью равномерного распределения, $w_{ij} = \exp(-v_{ij})$, $S_{ij} = \exp(-T_{ij})$.
\[
\begin{cases}
w_{ij} \sim \dUnif[0, 1] \\
S_{ia} = w_{ia}^{1/\exp(x_i^T \beta_a)}, \text{ где } \beta_a = 0 \\
S_{ib} = w_{ib}^{1/\exp(x_i^T \beta_b)} \\
S_{ic} = w_{ic}^{1/\exp(x_i^T \beta_c)} \\
y_i = \begin{cases}
a, \text{ если } S_{ia} = \max\{S_{ia}, S_{ib}, S_{ic}\} \\
b, \text{ если } S_{ib} = \max\{S_{ia}, S_{ib}, S_{ic}\} \\
c, \text{ если } S_{ic} = \max\{S_{ia}, S_{ib}, S_{ic}\} \\
\end{cases} \\
\end{cases}
\]

\section*{Бинарная логит-модель с помощью Гумбеля и логистического}

Бинарную логит модель обычно рассказывают в терминах логистического распределения:
\[
\begin{cases}
u_i \sim \dLogistic \\
y_i^* = x_i^T \beta + u_i \\
y_i =  \begin{cases}
0, \text{ если } y_i^* < 0 \\
1, \text{ если } y_i^* \geq 0 \\
\end{cases}
\end{cases}
\]

Можно изложить её и в терминах Гумбеля
\[
\begin{cases}
u_{ij} \sim \dGumbel \\
y_{ia}^* = x_i^T \beta_a + u_{ia}, \text{ где } \beta_a = 0 \\
y_{ib}^* = x_i^T \beta_b + u_{ib} \\
y_i = \begin{cases}
a, \text{ если } y_{ia}^* = \max\{y_{ia}^*, y_{ib}^*\} \\
b, \text{ если } y_{ib}^* = \max\{y_{ia}^*, y_{ib}^*\} \\
\end{cases}
\end{cases}
\]
Замечаем, что изложение в терминах Гумбеля переводится в классическое с логистическим распределением связкой
\[
y_i^* =y_{ib}^* - y_{ia}^* =  x_i^T \beta_b + u_{ib} - u_{ia}.
\]

В узких кругах широко известно, что разница двух независимых распределений Гумбеля $u_{ib} - u_{ia}$ имеет логистическое распределения.
Тут пруф :)


\section*{Первое событие в конкурирующих потоках}

Представим себе дюжину $3 + 4 + 5 = 12$ независимых пуассоновских потоках единичной интенсивности. 
На первые три потока поставим букву $A$, на следующие четыре потока — букву $B$, на следующие пять потоков — букву $C$.

Зададимся вопросом, на какую букву придётся самое первое происшествие в этих $12$ потоках?
В силу симметрии, первое происшествие ложится равновероятно на каждый исходный из дюжины потоков. 
Поэтому ответ имеет простой вид
\[
\P(A) = \frac{3}{3 + 4 + 5}, \quad \P(B) = \frac{4}{3 + 4 + 5}, \quad \P(C) = \frac{5}{3 + 4 + 5}.
\]

Теперь обратим внимание, что $A$-потоки можно сложить и получить один поток с интенсивностью три
происшествия в час, $B$-потоки можно сложить в один поток с интенсивностью четыре,
и $C$-потоки можно сложить в один поток с интенсивностью пять. 
Свойство суммирования для пуассоновского потока можно доказать, например, сложив независимые пуассоновские случайные величины.
Также можно построить доказательство, опираясь на аксиомы пуассоновского потока и математическое ожидание числа событий за единицу времени.

И мы видим следующий факт:

\begin{tcolorbox}[colback=yellow!50!red!25!white]
Если $k$ независимых пуассоновских потоков имеют интенсивности $\lambda_1$, $\lambda_2$, \dots, $\lambda_k$,
а $T_j$ — время первого происшествия в $j$-м потоке, то
\[
\P(T_j = \min\{T_1, T_2, \dots, T_k\}) = \frac{\lambda_j}{\lambda_1 + \lambda_2 + \dots + \lambda_k}.
\]
Другими словами, вероятность того, что на $j$-й поток придётся первое происшествие пропорциональна его интенсивности.
\end{tcolorbox}
    

\section*{Мультиномиальная логит-модель}

Рассмотрим некоторого индивида и три альтернативы, $A$, $B$ и $C$. 
Представим себе мысленный эксперимент. 
Индивид изготавливает три кастрюли в которые капли дождя попадают с интенсивностями $\lambda_A = \exp(V_A)$, 
$\lambda_B = \exp(V_B)$, $\lambda_C = \exp(V_C)$.
Можно считать, что $\lambda_j$ — это площадь поверхности кастрюли, а индивид выберет ту альтернативу, куда попадёт первая капля дождя.

Поток дождя, попадающий в каждую из кастрюль — это пуассоновский поток :)

Вероятности выбора альтернатив равны 
\[
\P(A) = \frac{\lambda_A}{\lambda_A + \lambda_B + \lambda_C}, \dots
\]
Время попадания первой капли в $j$-ю кастрюлю $T_j$ имеет экспоненциальное распределение с параметром $\lambda_j$.
Отсюда, $\lambda_j T_j \sim \dExpo(1)$ и $u_j = -\ln (\lambda_j T_j) \sim \dGumbel$,
\[
u_j = -V_j - \ln T_j \sim \dGumbel
\]
Назовём величину $y_j^* = -\ln T_j$ полезностью от альтернативы $j$,
\[
y_j^* = -\ln T_j = V_j + u_j
\]
Индивид выбирает ту альтернативу, у которой полезность $y_j^*$ больше!
По-доказанному,
\begin{align*}
\P(y_j^* = \max\{y_1^*, \dots, y_k^*\}) = \P(T_j = \min\{T_1, T_2, \dots, T_k\}) = \\
= \frac{\lambda_j}{\lambda_1 + \lambda_2 + \dots + \lambda_k} = 
\frac{\exp(V_j)}{\exp(V_1) + \exp(V_2) + \dots + \exp(V_k)}
\end{align*}

\section*{Классическое доказательство}

Для полноты картины приведём классическое доказательство формулы для вероятностей в мультиномиальной логит-модели напрямую из распределения Гумбеля. 
...



\section{Источники мудрости}

\printbibliography

\end{document}

